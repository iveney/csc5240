\documentclass[a4paper,12pt]{article}
\usepackage{times}
\usepackage[pdftex]{graphicx,color}
\usepackage[colorlinks=false,pdfborder=000]{hyperref}
\usepackage{enumerate}
\usepackage{multirow}
\usepackage{titling}
\usepackage[top=1.2in, bottom=1.2in, left=1.2in, right=1.2in]{geometry}
\usepackage{comment}
\usepackage{subfigure}
\usepackage{tikz}
\usepackage{amsmath}
\usetikzlibrary {snakes,arrows}
%%%%%%%%%%%%%%%%%%%%%%%%%%%%%%%%%%%%%%%%%%%%%%%%%%%%%%%%%%%%%%%%%%%%%%%%%%%%%%%%%%%%%%%%%%%%%%%%%%%%%%%%%%%%%%%%%%%%%%%%%%%%
\newcommand{\CSE}{\href{http://www.cse.cuhk.edu.hk}{Department of Computer Science and
Engineering}}
\newcommand{\CUHK}{\href{http://www.cuhk.edu.hk}{The Chinese University of Hong Kong}}
\newcommand{\mymail}{\mbox{\textcolor{blue}{\underline{zgxiao@cse.cuhk.edu.hk}}}}
\newcommand{\myname}{\href{http://www.cse.cuhk.edu.hk/~zgxiao}{XIAO Zigang}}
\newcommand{\header}[1]{\noindent {\bf \\#1\\}}
\newcommand{\NE}{Nash Equilibrium }
\newcommand{\tot}{\frac{2}{3}}
%%%%%%%%%%%%%%%%%%%%%%%%%%%%%%%%%%%%%%%%%%%%%%%%%%%%%%%%%%%%%%%%%%

% modify the font size of title, author and date
\pretitle{\begin{center}\bf \Huge} \posttitle{\par\end{center}}
\preauthor{\begin{center}
            \small \small \lineskip 0.5em%
            \begin{tabular}[t]{c}}
\postauthor{\end{tabular}\par\end{center}}
\predate{\begin{center}\small} \postdate{\par\end{center}}

\title{CSC5240 Project Proposal\\\Large Droplet Routing and Manipulation for Cross-Referencing Digital Microfluidic Biochip}
\author{\myname\\\mymail\\\CSE\\\CUHK}
\date{\today}

%%%%%%%%%%%%%%%%%%%%%%%%%%%%%%%%%%%%%%%%%%%%%%%%%%%%%%%%%%%%%%%%%%
\begin{document}
\bibliographystyle{unsrt}
\maketitle
\abstract{Digital Microfluidic Biochip (DMFB) has drawn lots of attention nowadays,
 especially the \textit{cross-referencing biochip}.
Due to electrode constraint, the manipulation of droplets to avoid electrode interference becomes crucial in droplet routing.
This project tries to solve this problem by using constraint satisfaction technique.
}
%%%%%%%%%%%%%%%%%%%%%%%%%%%%%%%%%%%%%%%%%%%%%%%%%%%%%%%%%%%%%%%%%%
\section{Background}
Digital Microfluidic Biochip has been receiving more and more attention nowadays.
It is crucial in medical, pharmaceutical and environmental monitoring applications \cite{verpoorte2003mmm}.
The discrete objects we want to operate on chip are called \textit{droplets}.
Analogous to VLSI synthesis, a top-down approach can be applied to biochip design,
and \textit{droplet routing} is one of the important steps in biochip synthesis.
The goal is to route a set of droplets from their \textit{source} to their \textit{sink}.

Currently, there exist two kinds of techniques to manipulate droplets movement.
The first one is \textit{direct-addressing},
in which an electrode is fixed under each cell of the chip.
It is simple, yet expensive and rather limited for large scale product.
Another more promising method is \textit{cross-referencing},
which applies high/low voltage to a row electrode and a column electrode
in order to activate the cell in their intersection point.
However, because the activation of a cell is in a row-column manner,
which means \textit{extra cells} may be activated when multiple rows/columns are addressed.
We call it \textit{electrode interference} and it imposes \textit{electrode constraint} on droplet routing.
The objective of \textit{Droplet Routing and Manipulation in Cross-Referencing Digital Microfluidic Biochip}
is to route the droplets and minimize the droplet routing time while not violating the electrode constraint.

\section{Literature Review}
The earliest article about droplet routing problem can be found in \cite{bohringer2003osm},
in which the author used a \textit{robotic} viewpoint to solve routing.
\cite{bohringer2006mac}\cite{1326232}\cite{1353672} use \textit{$A*$ search,network flow} and \textit{bypassibility} heuristic to generate routes.
However, none of them considered cross-referencing.
Cross-referencing routing is first considered in \cite{bohringer2003osm},but no concrete algorithm is given.
Currently there are only three papers handle droplet routing and cross-referencing manipulation at the same time.
\cite{griffith2006pcr} reduces the problem to a \textit{graph-coloring} problem.
\cite{1266484} transform the problem to a \textit{maximum clique partition} problem.
Both of them adapts the 1st approach which will be mentioned later.
\cite{yuh2008pib} first deals with the problem using 2nd approach in next section.
More specifically, it uses a \textit{progressive linear programming} with heuristic at each step to get a near-optimal solution.

\section{Problem Formulation}
There is a \text{route} for each droplet. Nevertheless,
these routes are \textit{virtual routes} which are different from those in VLSI design,
since they can \textit{intersect} with each other,
and we only need to ensure the droplets would not ``crash'' each other at the intersection point.

Each droplet can be interpreted as a moving robot.
Then the routing problem can be reduced to \textit{motion planning} problem in robotics research.
It is proved to be NP-hard by Erdmann et al.\cite{erdmannmovingobj}.

Moreover, because of \textit{electrode interference},
it is almost impossible to accomplish the parallel moving of droplet, which can be easily achieved in direct-addressing biochip.
In this project we are going to explore the possibility of solving this problem by using \textit{constraint programming} technique.

\section{Implementation Plan}
There are at least two different approaches to attack this problem.
\begin{enumerate}
  \item View droplet routing and droplet manipulation as two sequential process.
  The former step's output is the input for the latter step.
  \item Consider droplet routing while at the same time avoiding electrode interference, i.e.
  we treat droplet routing and droplet manipulation as a singular process.
\end{enumerate}

It is intuitive and relative easier to implement the 1st approach.
However, due to the independence of two process, we can not guarantee to get a \textit{optimal result}.
We have to \textit{schedule} the droplets' movement at each time step.
We may even fail in the manipulation step, posing the need to adjust original routes.
In the 2nd approach, we can ensure that the routes can be generated optimally.
But without decomposition of two process, the problem becomes much harder.

In this project, both implementation will be tried, though at last maybe only one of them can be feasibly implemented.
%%%%%%%%%%%%%%%%%%%%%%%%%%%%%%%%%%%%%%%%%%%%%%%%%%%%%%%%%%%%%%%%%%
%%%%%%% End of Document
\bibliography{proposal}
\end{document}
